\chapter*{Conclusions}
Using arrays of Transition Edge Sensors, the Holmes experiment has succesfully measured the calorimetric EC decay
spectrum of $^{163}$Ho, paving the way towards its first measurement of the electronic neutrino mass.
In order to support the upcoming experimental efforts, this thesis presented two Bayesian data analysis
applications focused on different parts of the energy spectrum measured by Holmes.

The first analysis involved a preliminary study of the M1 peak aimed at characterizing its position and width from
experimental data. An approximate form of the spectrum was used in the fit, considering only a Lorentzian and
first-order polynomial multiplied with a fixed phase-space and convolved with a Gaussian experimental response. After
assessing the robustness of the basic model, Bayesian model comparison was applied in order to select an alternative
peak profile describing a small asymmetry in the measured lineshape, with the results favoring an energy-dependent width.

One of the main challenges of the data analysis for Holmes consists in accurately considering the data from many
detectors with different systematics. In the characterization of the peak, the main obstacle to the determination of
\(E_{M1}\) is the systematic introduced by energy calibration, which results in a different acquired spectrum for
each TES. Considering the calibration systematic of each detector as an overall shift \(\delta E\) in the measured
energy, and assuming that the distribution of \(\delta E\) is dispersed around 0, a global fit on the data acquired by
16 detectors was performed by constructing a hierarchical model that allowed estimating all \(\delta E\)s and the
parameters of the peak at the same time. The outcome is a preliminary estimate of \(E_{M1} = 2043.1 \pm 1.8\) eV and
\(\Gamma_{M1} = 13.8 \pm 0.1\) eV, with both values being within $2\sigma$ from previous experimental results
by the Echo collaboration. Moreover, a small but nonzero asymmetry \(\alpha = -9.8 \times 10^{-3} \pm 0.6 \times 10^{-3}\) was found, hinting at the possibility of unresolved higher-order peaks at energies lower than \(E_{M1}\).

%Finally, a mean energy resolution of \(FWHM = 7.5 \pm 1.2\) eV and a dispersion in energy calibration of \(\sigma_{E} = 7.1 \pm 1.3\) eV were measured in the global fit.

The second part of the data analysis instead focused on the endpoint region, with the aim of providing an estimate of the upper limit on \(m_{\nu_{e}}\) which could be achieved by the current setup of Holmes in the near future. Considering the current energy resolution and expected background fraction, this analysis focused on a three months measurement involving 64 detectors implanted with 1 Bq of \(^{163}\)Ho each.

The data was obtained from a Monte Carlo simulation of the first order spectrum, initially considering all of the detectors to be identical and
fitting only their combined spectrum. This allowed establishing a basic model involving a small number of parameters
and testing its robustness with methods such as Simulation-Based-Calibration and prior sensitivity analysis. The latter
revealed that, for the low available statistics, the fit does not affect the energy resolution parameter \(FWHM\),
which can be set to a fixed reasonable estimate without altering the posterior for \(m_{\nu_{e}}\).

Having assessed the robustness to the priors, it was observed that fluctuations in the data could produce varying
estimates of \(m_{\nu_{e}}\). This required constructing a "robust posterior" by repeatedly generating new data,
refitting the model, and combining all the resulting samples. While simple, this method allowed obtaining more conservative estimates that consider the additional variability of the data.

To provide a more realistic limit, a more complex dataset involving 64 different detectors was simulated, varying normally their activity, energy
resolution, and calibration systematics within their observed ranges. The resulting data was fitted with a large model
constructed by extending the basic model to consider separate spectra from different detectors and employing the same hierarchical
approach used in the peak characterization to introduce the calibration systematics. Due to the low number of events in
each detector, the global fit is not able to obtain estimates of the detector-specific
systematics significantly different from their priors, and the resulting limit on \(m_{\nu_{e}}\) is sligthly worse than that obtained by fitting the combined different spectra with the basic model.

Finally, the basic model was repeatedly used to fit the compound spectrum of different detectors, and the resulting robust posterior was compared to that obtained from identical detectors.
While the penalty induced by neglecting the varying systematics doesn't worsen the upper limit on \(m_{\nu_{e}}\)
significantly, this result has only been proved under the assumption of normally distributed features between the
detectors, which could differ significantly from the experimental observations of the upcoming measurements. With these
considerations in mind, this analysis establishes $m_{\nu_{e}} \le 35$ eV at 90\% credibility level as a possible target for Holmes' sensitivity to the neutrino mass in the near future.
