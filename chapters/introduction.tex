\chapter*{Introduction}

The Standard Model of particle physics predicts the existence of left-handed neutrinos, which are part of
electroweak doublets alongside charged leptons. Neutrinos are unique in that they carry zero electric charge and lack
color charge, only interacting through the weak force. Notably, the Standard Model does not include right-handed neutrino components, which leads to the prediction that their left-handed counterparts should be massless.
Neutrinos come in three distinct flavors, associated with the electron, muon, and tau charged leptons in their
respective doublets. These flavor neutrinos can be observed when they are produced alongside or from specific charged
leptons, and can be identified with the interaction states.

A groundbreaking discovery in neutrino physics came in 1998 when the Super-Kamiokande collaboration and later the SNO
collaboration in 2001 confirmed the existence of neutrino flavor oscillations. This phenomenon consists in the fact that
as neutrinos propagate over varying distances and energies, their flavor can periodically change. Its discovery established that neutrinos do indeed possess mass, and their lepton flavors can mix.
Unlike heavy charged leptons and quarks, which quickly decay or hadronize, neutrinos 
can traverse long distances. This property allows to observe neutrino flavor oscillations, which are a
consequence not only of finite masses but also of a difference between the three mass eigenstates $\nu_{1,2,3}$ and the
flavor states $\nu_{e,\mu,\tau}$. 
The mixing between these states is described by a non-trivial Pontecorvo-Maki-Nakagawa-Sakata (PMNS) matrix, akin to the
Cabibbo-Kobayashi-Maskawa (CKM) matrix in the quark sector. The PMNS matrix can be parameterized in terms of three
mixing angles, denoted as $\theta_{ij}$, and one CP-violating phase $\delta_{CP}$:
\begin{equation}
U_{PMNS} = \begin{pmatrix}
1 & 0 & 0 \\
0 & c_{23} & s_{23} \\
0 & -s_{23} & c_{23}
\end{pmatrix}
\begin{pmatrix}
c_{13} & 0 & s_{13}e^{-i\delta_{CP}} \\
0 & 1 & 0 \\
-s_{13}e^{i\delta_{CP}} & 0 & c_{13}
\end{pmatrix}
\begin{pmatrix}
c_{12} & s_{12} & 0 \\
-s_{12} & c_{12} & 0 \\
0 & 0 & 1
\end{pmatrix},
\end{equation}

Where $c_{ij} = \cos\theta_{ij}$ and $s_{ij} = \sin\theta_{ij}$. Additionally, if neutrinos are their own antiparticles,
a property denoted as being a Majorana fermion, two more phases may exist, but they cannot be measured through
oscillation experiments. For a neutrino created in a flavor state $\alpha$, the probability $P(\nu_\alpha \rightarrow
\nu_\beta)$ of detecting it in a flavor state $\beta$ after traveling through vacuum is described by the following equation:

\begin{eqnarray}
      P({\nu_\alpha \rightarrow\nu_\beta}) = \delta_{\alpha\beta} - 4\sum_{j > i}\mathrm{Re}\left(U_{\alpha j}^*\,U_{\beta j}\,
      U_{\alpha i}\,U_{\beta i}^*\right)\sin^2\left(\frac{\Delta m^2_{ji}\,L}{4E}\right) \\ 
      + 2\sum_{j > i}\mathrm{Im}
      \left(U_{\alpha j}^*\,U_{\beta j}\,U_{\alpha i}\,U_{\beta i}^*\right)\sin\left(\frac{\Delta m^2_{ji}\,L}{2E}
      \right) \nonumber
\end{eqnarray}
Here, $E$ is the neutrino energy, $L$ is the source-detector distance, and $\Delta m^2_{ji} = m_j^2 - m_i^2$. The
second term is responsible for CP violation and is present only if $\delta_{CP}\neq 0$.

Neutrino oscillations are studied using both terrestrial and astrophysical sources, and the accessible
oscillation channels depend on the energy spectrum of the neutrinos. For instance, oscillations primarily due to the
$\theta_{23}$ angle were first observed in atmospheric neutrinos, produced in
cosmic-ray interactions in the Earth's atmosphere. On the other hand, solar neutrinos, generated in the core of the Sun through fusion processes, have been crucial for
constraining $\theta_{12}$ and $\Delta m^2_{21}$. 
Because of this, the mixing angles $\theta_{23}$ and $\theta_{12}$ and the corresponding square mass differences are
commonly referred to as describing atmospheric and solar neutrino oscillations, respectively. The remaining relatively small mixing angle $\theta_{13}$ approximately decouples atmospheric and solar oscillations. 
Reactor neutrinos, produced in the cores of nuclear reactors, have provided precise measurements of $\theta_{13}$,
$\theta_{12}$, and $|\Delta m^2_{31}|$. Finally, accelerator neutrino beams have been employed to constrain various mixing
parameters, depending on the selected beam. Global analyses combining data from various sources have
been used to determine the values of all mixing angles and $\Delta m^2_{21}$, while the CP-violating phase $\delta_{CP}$ and the sign of $\Delta m^2_{31}$ remain undetermined.
This unknown sign means that there are two pssible ways to order the mass states, a problem that is known as the
neutrino mass hierarchy problem. 
In the first configuration, known as the Normal Ordering (NO), the two lightest neutrino mass
eigenstates exhibit a minimal mass difference, approximately on the order of 10 meV, while the third eigenstate
possesses a mass roughly 50 meV higher. In the Inverted Ordering (IO), the lightest neutrino mass eigenstate is
succeeded by a doublet of higher mass eigenstates, with a small mass difference within the doublet. Current experimental
data exhibit a slight preference for the Normal Ordering.


The masses of neutrinos are remarkably smaller, at least five orders of magnitude, than those of any other fermions in
the Standard Model. This significant disparity hints at the possibility of a distinct mechanism responsible for
generating them. Due to oscillations only depending on the difference between neutrino masses, their observation can only provide a lower
bound on the absolute neutrino mass scale. A direct measurement of this parameter may not only shed light on many open problems,
such as the question of mass hierarchy, but also offer new insights into the fundamental question of how these particles acquire
mass. Moreover, neutrinos play a pivotal role in shaping the evolution of the cosmos on
a large scale, and a direct measurement of the neutrino mass could thus serve as a crucial input for models describing the formation of cosmological structures.

